\documentclass[11pt,ngerman]{article}
\usepackage[utf8]{inputenc}
%?
\usepackage[a4paper,left=2cm,right=2cm,top=2cm,bottom=2cm,bindingoffset=5mm]{geometry}
%?
\usepackage{hyperref}
\usepackage{booktabs}
\usepackage{graphicx}
\usepackage{enumitem}

\begin{document}
\title{Summary Computer Networks}
\author{John}
\maketitle

\tableofcontents



\section{Motivation}

\subsection{Communication	Metaphors}
\begin{itemize}[noitemsep]
	\item Phase 1: Person to person
	\item Phase 2: Person to machine
	\item Phase 3: Machine to machine/Network of computers
	\item Phase 4: The internet of Things
\end{itemize}

\subsection{History}
\begin{itemize}[noitemsep]
	\item 1837:	Samuel	Morse	develops	the	telegraph
	\item 1953:	First	transatlantic	Telephone	line
	\item 1876:	Alexander	Graham	Bell	patents	the	telephone	(tele=distant,	phone=voice)
\end{itemize}

\subsection{Telephone	Network}
Existing	networks	are	going	to	be	integrated\\
\includegraphics[width=5in]{images/Selection_001.png}

\subsection{The Internet}7
The internet consists of
\begin{itemize}[noitemsep]
	\item a	set	of	computers,	which
	\begin{itemize}[noitemsep]
		\item use	the	TCP/IP	protocols
		\item are	somehow	(directly	or	indirectly)	connected
		\item offer	or	use	particular	services
	\end{itemize}
	\item a	set	of	users,	which	have	access	to	these	services
	\item a	set	of	other	networks,	which	(somehow)	are	accessible
\end{itemize}

Design Principles
\begin{itemize}[noitemsep]
	\item Minimalism	and	autonomy  - The	network	operates	by	itself	, does	not	require	internal	changes	when	new	networks	are	added
	\item Best-effort	service	model
	\item Soft-state	(stateless) - The	routers	do	not need	to	maintain end-to-end	communication	
information
	\item Decentralization
\end{itemize}



%%%%%%%%%%%%%%%%%%%%%%%%%%%%%%%%%%%%%%%%%%%%%%%%%%%%%%%%%%



\section{Introduction}
\subsection{Data	Communication}
Data	communication	is	the	processing	and	the	transport	of	digital	data	over	
connections	between	computers	(generally	over	large	distances).\\
Data	communication	comprises	two areas: Computer	Networks and Communication	Protocols

\subsection{What	is	Digital	Data?}
\begin{itemize}[noitemsep]
	\item Data: Representation	of	facts in	a	formal	way, processable by humans and machines, e.g. a language
	\item Information:  is	whatever	contributes	
to	a	reduction	in	the	uncertainty	of	the	state	of	a	system, can only be handled by humans
	\item Signal: is	the	physical	representation	
of	data	by	spatial	or	timely	variation	
of	physical	characteristics
	\item Example:  Sounds	of	a	language	(Data)	during	speaking	are	acoustic	waves	(Signals)
\end{itemize}

\subsection{Data	Communication}
\begin{itemize}[noitemsep]
	\item Sharing	resources	saves	costs
	\item Exchange	of	information
\end{itemize}

\subsection{Networking	Principles}
Communication	Peers
\begin{itemize}[noitemsep]
	\item Unicast:	Two	communication	peers	
communicate	over	a	Point-to-Point	
connection.
	\item Multicast:	One	sender	
communicates	to	several	receivers,	
which	are	known.
	\item Broadcast:	One	sender	transmits	to	
all	other	peers.
Typically	the	other	peers	are	
(partially)	unknown.
	\item Others:	Anycast,	Geocast,	etc.
\end{itemize}

Transmission
\begin{itemize}[noitemsep]
	\item Serial Transmission
	\item Parallel Transmission (Problem: synchronisation of the data)
	\item Asynchronous Transmission:	Transmission	in	which	each	block	
(character)	is	individually	synchronized\\
\includegraphics[width=5in]{images/Selection_002.png}

	\item Synchronous Transmission:	Transmission	in	which	the	time	of	
occurrence	of	each	signal	representing	a	bit	is	related	to	a	fixed	time	
frame \\
\includegraphics[width=5in]{images/Selection_003.png}
\end{itemize}

Connection	Properties\\
\includegraphics[width=5in]{images/Selection_004.png}

Multiplexing:
Combining	multiple	data	
channels	into	a	single	data	
channel	at	the	source\\

Quality\\
\begin{itemize}[noitemsep]
	\item Technical	Performance (Delay-Bandwidth-Product	=	
Store	capacity of	the	line)
	\begin{itemize}[noitemsep]
		\item Delay	[s]
		\item Jitter	[s]
		\item Throughput	[bit/s]
		\item Data	rate	[bit/s] (wird vorgegeben)
	\end{itemize}
	\item Costs
	\item Reliability
	\item Security	and	Protection\\
		Safety measures: Encryption, Trustworthy	systems	\\
	\includegraphics[width=5in]{images/Selection_005.png}
\end{itemize}


The	Client/Server	Principle
\begin{itemize}[noitemsep]
	\item Client $\rightarrow$ Server: Request
	\item Server $\rightarrow$ Client: Reply 
\end{itemize}
\begin{itemize}[noitemsep]
	\item Advantages
	\begin{itemize}[noitemsep]  
	\item Cost	reduction
	\item Better	usage	of	resources
	\item Modular	extensions
	\item Reliability	by	redundancy
	 \end{itemize}
	\item Server: Program	(process)	which	offers	a	service	over	a	network.	
	\item Client: Program	(process)	which	uses	a	service	offered	by	a	server.
\end{itemize}


Peer-to-Peer	Principle (ursprüngliche Kommunikation im Internet)
\begin{itemize}[noitemsep]
	\item Equal	partners,	no	fixed	client	and	server	roles
	\item Connections	between	any	pair	of	computers
	\item Establishment	of	a	whole	network	of	connections
	\item Best	example:	File	Sharing,	e.g.,	Napster,	Gnutella
\end{itemize}

\subsection{Communication	Protocols}

A	protocol	is	the	set	of	agreements	between	(application)	processes	with	the	purpose	of	
communication.\\
To	enable	understanding	in	communication,	all	communication	partners	have	to	speak	the	
same	language.
\begin{itemize}[noitemsep]
\item Data	formats	and	their	semantics
\item Control	over	media	access
\item Priorities
\item Handling	of	transmission	errors
\item Sequence	control
\item Flow	control	mechanisms
\item Segmentation	and	composition	of	long	messages
\item Multiplexing
\item Routing
\end{itemize}
	\includegraphics[width=5in]{images/Selection_006.png}\\
	$\rightarrow$ communication between horizontal layers
	
Peer	of	a	Layer
\begin{itemize}[noitemsep]
\item
use	one	service	
(except	the	bottom)
\item
offer	a	service	
(except	the	top)
\item
do	not	need	to	know	other	
than	the	next	lower	one
\item
talk	according	to	the	rules
\end{itemize}

Communication	architectures	are	based	on
\begin{itemize}[noitemsep]
\item  Service	=	Communication	Service
\item Rules	=	Communication	Protocol

\end{itemize}
A	service	is	offered	from	a	service	provider at	a	service	interface	
to	service	users.\\
Types	of	services	are:
\begin{itemize}[noitemsep]
\item Request
\item Indication
\item Response
\item Confirmation
\end{itemize}

\includegraphics[width=5in]{images/Selection_007.png}\\

\textbf{Types of Services}
\begin{itemize}[noitemsep]
\item Unacknowledged	Service
\begin{itemize}[noitemsep]
\item Modeled	after	the	postal	service
\item Initiated	by	the	service	user
\end{itemize}
\item Acknowledged	Service (Transaction)
\item Connection-oriented	Service
\begin{itemize}[noitemsep]
\item Modeled	after	the	telephone	system
\item Before	the	instances	on	Layer-(N)	can	
exchange	data,	a	connection	on	
Layer-(N-1)	has	to	be	established
\item Negotiation	of	protocol	parameters\\
$\rightarrow$ Communication	context
\end{itemize}
\item Connectionless	Service
\begin{itemize}[noitemsep]
\item  Modeled	after	the	postal	service
\item No	establishment	of	connection	on	a	
lower	layer	required\\
$\rightarrow$ No	communication	context
\end{itemize}
\end{itemize}

\includegraphics[width=5in]{images/Selection_008.png}\\
\includegraphics[width=5in]{images/Selection_009.png}\\

\textbf{Service primitives}\\
\begin{tabular}[t]{ll}
Primitive & Meaning\\
\hline
LISTEN & Block	waiting	for	an	incoming	connection\\
CONNECT & Establish	a	connection	with	a	waiting	peer\\
RECEIVE & Block	waiting	for	an	incoming	message\\
SEND & Send	a	message	to	the	peer\\
DISCONNECT & Terminate	a	connection\\
\end{tabular}\\
\includegraphics[width=3in]{images/Selection_010.png}\\

\subsection{ISO/OSI Reference Model}
\includegraphics[width=5in]{images/Selection_011.png}\\

\begin{enumerate}[noitemsep]
\item Physical	Layer
	\begin{itemize}[noitemsep]
	\item Responsible for single bit transmission
	\item Details are defined: type	of	cables,	meaning	of	pins	of	network	
	connectors,	transmission	direction	on	the	cable
	\end{itemize}
	
\item Data Link Layer
	\begin{itemize}[noitemsep]
	\item Ensures	an	error-free data	transmission	between	two	directly	connected	
devices $\rightarrow$ segmented into frames (transmitted separately)
	\item Receiver checks the correctness (checksum)
	\item flow	control	is	used	to	control	the	re-transmission	of	corrupt	
frames	and	protect	the	receiver	from	overload.
	\item control	of	medium	access (prevent address conflict)
	\end{itemize}

\item Network Layer
	\begin{itemize}[noitemsep]
	\item Data-transmission over	large	distances	and	
between	heterogeneous	sub-networks
	\item uniform	addressing	of	hosts
	\item 	routing:	select	a	path	through	the	network.		
	\item Quality	of	Service	(QoS)	issues,	i.e.,	if	too	many	packets	are	present	at	the	
same	time	in	the	network,	they	may	form	bottlenecks. (congestion, maximum size	of	the	transferred	data	units	(MTU), delay,	jitter,	transit	time,	etc.)
	\end{itemize}
	
\item Transport Layer
	\begin{itemize}[noitemsep]
	\item end-to-end	communication between	two	processes
	\item Ensure	that	the	data	are	receipt	complete and	in	correct	order
	\item current	network	state	is	monitored	to	adapt	to	the	
receiver	and	to	the	network	capacity
	\end{itemize}
	
\item Session Layer
	\begin{itemize}[noitemsep]
	\item manages	reliable	data	transport
	\item offers dialogue	
control,	i.e.,	define	the	direction	of	the	transmission.
	\item token	management (allows operation)
	\item set synchronization	points in the communication process
	\end{itemize}
	
\item Presentation	Layer
	\begin{itemize}[noitemsep]
	\item Represent	the	data	to	be	transmitted	in	a	way,	that	they	can	be	handled	from	
different	computer	systems	
	\item Data	are	encoded in	an	abstract (and	commonly	recognized)	data	format	before	
the	transmission	and	are	coded	back	by	the	receiver	into	its	own	data	format.
	\end{itemize}
	
\item Application	Layer
	\begin{itemize}[noitemsep]
	\item standard protocols	are	provided,	that	can	be	used	from	
applications
	\item interface	to	file	transfer
	\end{itemize}
\end{enumerate}

\textbf{Interplay	between	the	Layers
}\begin{itemize}[noitemsep]
\item Layer	(N-1)	offers	its	functionality	to	layer	N	as	a	communication	service.
\item Layer N	enhances	the	data	to	be	sent	with	control	information	(Header)	and	sends	the	data	
together	with	the	header	as Protocol	Data	Units (PDU).
\item Depending on the protocol, N-PDUs can be segmented into several
(N-1)-PDUs before transmission
\end{itemize}

\subsection{The	TCP/IP	Reference	Model}
\begin{enumerate}
	\item Host-to-Network	Layer (1+2)
		\begin{itemize}[noitemsep]
		\item Not defined exactly
		\item host	
must	be	connected	to	the	network	via	a	protocol	in	a	way	that	it	is	able	to	send	
and	receive	IP	datagrams
		\end{itemize}
	\item Internet	Layer (3)
		\begin{itemize}[noitemsep]
		\item 	interworking	of	different	networks
		\item enables	communication	between	hosts	over	the	own	network	
borders (	transmission	is	connectionless)
		\item  Router takes over the forwarding
		\item Path can be dynamic
		\item In	contrast	to	ISO,	only	one	packet	format	is	defined,	together	with	a	
connectionless	protocol,	the	Internet	Protocol	(IP)
		\end{itemize}
	\item Transport	Layer (4)
		\begin{itemize}[noitemsep]
		\item covers	the	communication	between	the	end	systems
		\item TCP (Transmission	Control	Protocol)	is	a	reliable,	connection-oriented protocol	
for	the	transmission	of	a	byte	stream between	two	hosts.
		\item UDP (User	Datagram	Protocol)	is	an	unreliable and	connectionless	protocol	
(best	effort).
		\end{itemize}
	\item Application	Layer (7)
		\begin{itemize}[noitemsep]
		\item defines	common	communication	services (HTTP, FTP,..)
		\end{itemize}
\end{enumerate}

\subsection{OSI	vs.	TCP/IP}

\begin{enumerate}[noitemsep]
	\item Time -
The	TCP/IP	protocols	were	already	widely	used	before	OSI	had	finished	the	
standardization	activities.
	\item Freedom	from	obligation (defines what not how) $\rightarrow$ incompatibility	of	products
	\item Complicatedness
	\item 	Political	reasons (Europe)
	\item Hurriedly	product	implementation
\end{enumerate}



\subsection{Standardization}
Two	types	of	standards
\begin{itemize}[noitemsep]
\item De	facto	standards
\item De	jure	standards
\end{itemize}

Organisationen:
\begin{itemize}[noitemsep]
\item ISO
\item Internet	Engineering	Task	Force
\item Institute	of	Electrical	and	Electronic	Engineers	(IEEE)
\end{itemize}

\subsection{Evolution	of	Computer	Networks}
\begin{itemize}[noitemsep]
\item First generation: via mainframe in computer center
\item Connection via LAN, router $\rightarrow$ rest of the world
\item computer centers via router connected to backbone $\rightarrow$ rest of the world
\end{itemize}

\subsection{Classification	of	Computer	Networks}
\begin{itemize}[noitemsep]
\item Personal	Area	Network	(PAN) - 1m
\item Local	Area	Network	(LAN) - 10-100m
\item Metropolitan	Area	Network	(MAN) - 1-10km
\item Wide	Area	Network	(WAN) - 100-1000km
\item Internet - 10000km 
\end{itemize}


%%%%%%%%%%%%%%%%%%%%%%%%%%%%%%%%%%%%%%%%%%%%%%%%%%%%%%%%%%%%%%%%%%%%%%%%%%%%%%%%%%%%%

\section{Physical Layer}

\subsection{Theoretical	Basis	for	Data	Communication}
\begin{itemize}[noitemsep]
\item Spectrum of	a	signal	is	the	range	of	
frequencies	it	contains
\item The	absolute	bandwidth of	the	signal	
is	the	width	of	the	spectrum
\item Bandwith of a medium: Frequency	range	which	can	be	
transmitted	over	a	medium
\end{itemize}

Transmission	of	information	can	take	place	on	
\begin{itemize}[noitemsep]
\item Baseband (information	is	transmitted	over	the	medium	as	it	is)\\
	$\rightarrow$ discrete	(digital) signals
\item Broadband (The	information	is	transmitted	analogous by modulating onto a carrier signal)\\
	used in optical and radio networks\\
	$\rightarrow$ continuous	(analogous)	signals
\end{itemize}

Nyquist- und	Shannon-Theorem\\
max.	data	rate	=	$2 \cdot B \cdot log_2 (n)$ vs. $B \cdot log_2 (1+SNR)$

\subsection{Analog	Data	and	Digital	Signals}

Pulse	Code	Modulation	(PCM)	is	based	on	the	sampling	theorem	
by	Shannon	and	Raabe: If	a	signal	is	sampled	at	regular	intervals	of	time	and	at	a	rate	higher	than	twice	the	highest	
significant	signal	frequency,	then	the	samples	contain	all	the	information	of	the	original	
signal.

\subsection{Data	Encoding}
see exercise
\subsection{Transmission	Media}
see exercise
\subsection{The	Last	Mile	Problem}
connect	private	homes	to	
the	Internet	without	installing	many	new	cables $\rightarrow$ Use	existing	telephone	lines:	re-use	them	for	data	
traffic\\

Examples: Modem, ISDN, DSL\\

Modulation:
\begin{itemize}[noitemsep]
\item Amplitude Modulation
\item Frequency Modulation
\item Phase Modulation
\end{itemize}

\subsection{Multiplexing}
Sharing	of	an	expensive	resource,	e.g.,	transmit	multiple	connections	over	
the	same	line\\
Frequency	Division	Multiplexing, Time	Division	Multiplexing

\subsection{Digital	Subscriber	Line	(DSL)}
Combination	of	usual	phone	service	
(analog/ISDN)	and	data	service: 
simply	use	the	whole	spectrum	a	copper	
cable	can	transfer,	not	only	the	range	up	to	
3.4	kHz!


\section{Data Link Layer}

\begin{itemize}[noitemsep]
\item rovides	a	well-defined	service	
interface	to	the	network	layer
\item deals	with	transmission	errors
regulates
the	flow	of	data	, 
the	access	to	the	medium, 
that	a	slow	receiver	is	not	swamped	by	
a	fast	sender
\end{itemize}

Parts:
\begin{itemize}[noitemsep]
\item Logical	Link	Control	(LLC)
\begin{itemize}[noitemsep]
\item  Organization	of	the	data	to	be	sent	into	frames
\item Guarantee	(if	possible)	an	error	free	transmission	between	neighboring nodes	by	...
\item Detection	(and	recovery)	of	transfer	errors
\item Flow	Control	(avoidance	of	overloading	the	receiver)
\item Buffer	Management
\end{itemize}
\item Medium	Access	Control	(MAC)
\begin{itemize}[noitemsep]
\item Access	control	to	the	communication	channel	in	broadcast	networks
\end{itemize}
\end{itemize}

\subsection{Error	Detection	and	Correction}
Compute	a	short	checksum	of	the	data	and	send	it	together	with	
the	data	to	the	receiver.	

CRC, Hamming Code $\rightarrow$ exercise

\subsection{Elementary	Data	Link	Protocols}

\includegraphics[width=5in]{images/Selection_016.png}

\textbf{Sliding Window}: Allow	sender	to	transmit	up	to	W
frames	before	blocking\\
 $\rightarrow$ exercise\\
\includegraphics[width=5in]{images/Selection_017.png}

\subsection{High	Level	Data	Link	Control	(HDLC)}
Three	types	of	stations
\begin{itemize}[noitemsep]
\item Primary	station:	responsible	for	controlling	the	operation	of	the	link.	
\item Secondary	station:	operates	under	the	control	of	the	primary	station.	
\item Combined	station
\end{itemize}

\textbf{Bit stuffing:} Sender	inserts	a	zero	after	each	sequence	of	five	ones.	The	receiver	removes	this	
zero.\\
frame: 01111110 Address Control Data Checksum 01111110 \\
Control: 0 Seq P/F Next

\subsection{Point-to-Point	Protocol	(PPP)}
Establish	a	direct	connection	between	two	nodes\\
Features:
\begin{itemize}[noitemsep]
\item Framing	method	with	error	detection
\item Link	Control	Protocol
\item is	character	oriented	and	uses	byte-stuffing
\end{itemize}

\subsection{Protocol Verification}
\begin{itemize}[noitemsep]
\item Finite	state	machines
\item Petry nets
\end{itemize}

\end{document}
