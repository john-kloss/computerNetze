\documentclass[11pt,ngerman]{article}
\usepackage[utf8]{inputenc}
%?
\setlength{\parskip}{0.1cm}
\setlength{\baselineskip}{0.1cm}
\setlength{\parindent}{0em}
\usepackage[a4paper,left=2cm,right=2cm,top=2cm,bottom=2cm,bindingoffset=5mm]{geometry}
%?
\usepackage{hyperref}
\usepackage{booktabs}
\usepackage{graphicx}

\begin{document}
\title{Summary Computer Networks}
\author{John}
\maketitle

\tableofcontents



\section{Motivation}

\subsection{Communication	Metaphors}
\begin{itemize}
	\item Phase 1: Person to person
	\item Phase 2: Person to machine
	\item Phase 3: Machine to machine/Network of computers
	\item Phase 4: The internet of Things
\end{itemize}

\subsection{History}
\begin{itemize}
	\item 1837:	Samuel	Morse	develops	the	telegraph
	\item 1953:	First	transatlantic	Telephone	line
	\item 1876:	Alexander	Graham	Bell	patents	the	telephone	(tele=distant,	phone=voice)
\end{itemize}

\subsection{Telephone	Network}
Existing	networks	are	going	to	be	integrated\\
\includegraphics[width=5in]{images/Selection_001.png}

\subsection{The Internet}7
The internet consists of
\begin{itemize}
	\item a	set	of	computers,	which
	\begin{itemize}
		\item use	the	TCP/IP	protocols
		\item are	somehow	(directly	or	indirectly)	connected
		\item offer	or	use	particular	services
	\end{itemize}
	\item a	set	of	users,	which	have	access	to	these	services
	\item a	set	of	other	networks,	which	(somehow)	are	accessible
\end{itemize}

Design Principles
\begin{itemize}
	\item Minimalism	and	autonomy  - The	network	operates	by	itself	, does	not	require	internal	changes	when	new	networks	are	added
	\item Best-effort	service	model
	\item Soft-state	(stateless) - The	routers	do	not need	to	maintain end-to-end	communication	
information
	\item Decentralization
\end{itemize}



%%%%%%%%%%%%%%%%%%%%%%%%%%%%%%%%%%%%%%%%%%%%%%%%%%%%%%%%%%



\section{Introduction}
\subsection{Data	Communication}
Data	communication	is	the	processing	and	the	transport	of	digital	data	over	
connections	between	computers	(generally	over	large	distances).\\
Data	communication	comprises	two areas: Computer	Networks and Communication	Protocols

\subsection{What	is	Digital	Data?}
\begin{itemize}
	\item Data: Representation	of	facts in	a	formal	way, processable by humans and machines, e.g. a language
	\item Information:  is	whatever	contributes	
to	a	reduction	in	the	uncertainty	of	the	state	of	a	system, can only be handled by humans
	\item Signal: is	the	physical	representation	
of	data	by	spatial	or	timely	variation	
of	physical	characteristics
	\item Example:  Sounds	of	a	language	(Data)	during	speaking	are	acoustic	waves	(Signals)
\end{itemize}

\subsection{Data	Communication}
\begin{itemize}
	\item Sharing	resources	saves	costs
	\item Exchange	of	information
\end{itemize}

\subsection{Networking	Principles}
Communication	Peers
\begin{itemize}
	\item Unicast:	Two	communication	peers	
communicate	over	a	Point-to-Point	
connection.
	\item Multicast:	One	sender	
communicates	to	several	receivers,	
which	are	known.
	\item Broadcast:	One	sender	transmits	to	
all	other	peers.
Typically	the	other	peers	are	
(partially)	unknown.
	\item Others:	Anycast,	Geocast,	etc.
\end{itemize}

Transmission
\begin{itemize}
	\item Serial Transmission
	\item Parallel Transmission (Problem: synchronisation of the data)
	\item Asynchronous Transmission:	Transmission	in	which	each	block	
(character)	is	individually	synchronized\\
\includegraphics[width=5in]{images/Selection_002.png}

	\item Synchronous Transmission:	Transmission	in	which	the	time	of	
occurrence	of	each	signal	representing	a	bit	is	related	to	a	fixed	time	
frame \\
\includegraphics[width=5in]{images/Selection_003.png}
\end{itemize}

Connection	Properties\\
\includegraphics[width=5in]{images/Selection_004.png}

Multiplexing:
Combining	multiple	data	
channels	into	a	single	data	
channel	at	the	source\\

Quality\\
\begin{itemize}
	\item Technical	Performance (Delay-Bandwidth-Product	=	
Store	capacity of	the	line)
	\begin{itemize}
		\item Delay	[s]
		\item Jitter	[s]
		\item Throughput	[bit/s]
		\item Data	rate	[bit/s] (wird vorgegeben)
	\end{itemize}
	\item Costs
	\item Reliability
	\item Security	and	Protection\\
		Safety measures: Encryption, Trustworthy	systems	\\
	\includegraphics[width=5in]{images/Selection_005.png}
\end{itemize}


The	Client/Server	Principle
\begin{itemize}
	\item Client $\rightarrow$ Server: Request
	\item Server $\rightarrow$ Client: Reply 
\end{itemize}
\begin{itemize}
	\item Advantages
	\begin{itemize}  
	\item Cost	reduction
	\item Better	usage	of	resources
	\item Modular	extensions
	\item Reliability	by	redundancy
	 \end{itemize}
	\item Server: Program	(process)	which	offers	a	service	over	a	network.	
	\item Client: Program	(process)	which	uses	a	service	offered	by	a	server.
\end{itemize}


Peer-to-Peer	Principle (ursprüngliche Kommunikation im Internet)
\begin{itemize}
	\item Equal	partners,	no	fixed	client	and	server	roles
	\item Connections	between	any	pair	of	computers
	\item Establishment	of	a	whole	network	of	connections
	\item Best	example:	File	Sharing,	e.g.,	Napster,	Gnutella
\end{itemize}

\subsection{Communication	Protocols}

A	protocol	is	the	set	of	agreements	between	(application)	processes	with	the	purpose	of	
communication.\\
To	enable	understanding	in	communication,	all	communication	partners	have	to	speak	the	
same	language.
\begin{itemize}
\item Data	formats	and	their	semantics
\item Control	over	media	access
\item Priorities
\item Handling	of	transmission	errors
\item Sequence	control
\item Flow	control	mechanisms
\item Segmentation	and	composition	of	long	messages
\item Multiplexing
\item Routing
\end{itemize}
	\includegraphics[width=5in]{images/Selection_006.png}\\
	$\rightarrow$ communication between horizontal layers
	
Peer	of	a	Layer
\begin{itemize}
\item
use	one	service	
(except	the	bottom)
\item
offer	a	service	
(except	the	top)
\item
do	not	need	to	know	other	
than	the	next	lower	one
\item
talk	according	to	the	rules
\end{itemize}

Communication	architectures	are	based	on
\begin{itemize}
\item  Service	=	Communication	Service
\item Rules	=	Communication	Protocol

\end{itemize}
A	service	is	offered	from	a	service	provider at	a	service	interface	
to	service	users.\\
Types	of	services	are:
\begin{itemize}
\item Request
\item Indication
\item Response
\item Confirmation
\end{itemize}

\includegraphics[width=5in]{images/Selection_007.png}\\

\textbf{Types of Services}
\begin{itemize}
\item Unacknowledged	Service
\begin{itemize}
\item Modeled	after	the	postal	service
\item Initiated	by	the	service	user
\end{itemize}
\item Acknowledged	Service (Transaction)
\item Connection-oriented	Service
\begin{itemize}
\item Modeled	after	the	telephone	system
\item Before	the	instances	on	Layer-(N)	can	
exchange	data,	a	connection	on	
Layer-(N-1)	has	to	be	established
\item Negotiation	of	protocol	parameters\\
$\rightarrow$ Communication	context
\end{itemize}
\item Connectionless	Service
\begin{itemize}
\item  Modeled	after	the	postal	service
\item No	establishment	of	connection	on	a	
lower	layer	required\\
$\rightarrow$ No	communication	context
\end{itemize}
\end{itemize}

\includegraphics[width=5in]{images/Selection_008.png}\\
\includegraphics[width=5in]{images/Selection_009.png}\\

\textbf{Service primitives}\\
\begin{tabular}[t]{ll}
Primitive & Meaning\\
\hline
LISTEN & Block	waiting	for	an	incoming	connection\\
CONNECT & Establish	a	connection	with	a	waiting	peer\\
RECEIVE & Block	waiting	for	an	incoming	message\\
SEND & Send	a	message	to	the	peer\\
DISCONNECT & Terminate	a	connection\\
\end{tabular}\\
\includegraphics[width=3in]{images/Selection_010.png}\\

\subsection{ISO/OSI Reference Model}
\includegraphics[width=5in]{images/Selection_011.png}\\

\begin{enumerate}
\item Physical	Layer
	\begin{itemize}
	\item Responsible for single bit transmission
	\item Details are defined: type	of	cables,	meaning	of	pins	of	network	
	connectors,	transmission	direction	on	the	cable
	\end{itemize}
	
\item Data Link Layer
	\begin{itemize}
	\item Ensures	an	error-free data	transmission	between	two	directly	connected	
devices $\rightarrow$ segmented into frames (transmitted separately)
	\item Receiver checks the correctness (checksum)
	\item flow	control	is	used	to	control	the	re-transmission	of	corrupt	
frames	and	protect	the	receiver	from	overload.
	\item control	of	medium	access (prevent address conflict)
	\end{itemize}

\item Network Layer
	\begin{itemize}
	\item Data-transmission over	large	distances	and	
between	heterogeneous	sub-networks
	\item uniform	addressing	of	hosts
	\item 	routing:	select	a	path	through	the	network.		
	\item Quality	of	Service	(QoS)	issues,	i.e.,	if	too	many	packets	are	present	at	the	
same	time	in	the	network,	they	may	form	bottlenecks. (congestion, maximum size	of	the	transferred	data	units	(MTU), delay,	jitter,	transit	time,	etc.)
	\end{itemize}
	
\item Transport Layer
	\begin{itemize}
	\item end-to-end	communication between	two	processes
	\item Ensure	that	the	data	are	receipt	complete and	in	correct	order
	\item current	network	state	is	monitored	to	adapt	to	the	
receiver	and	to	the	network	capacity
	\end{itemize}
\end{enumerate}










\end{document}

Model	of	layers	is	applied	to	simplify	the	complexity
•
•
ISO/OSI
TCP/IP
There	are	many	global	players	in	computer	networking
•
Standardization
Computer	networks
•
Different	kinds	of	computer	networks	exist